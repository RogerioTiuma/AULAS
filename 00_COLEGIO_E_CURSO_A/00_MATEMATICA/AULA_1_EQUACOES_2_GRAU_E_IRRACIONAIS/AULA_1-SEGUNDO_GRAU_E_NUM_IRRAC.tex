\documentclass[12pt,a4paper]{article}
\usepackage[utf8]{inputenc}
\usepackage{amsmath}
\usepackage{amssymb}
\usepackage{geometry}
\geometry{margin=2.5cm}

\title{\textbf{Teoria e Lista de Exerc\'icios - Ensino Integrado IFPE}}
\author{Equa\c{c}\~oes do 2\textordfeminine\ Grau e N\'umeros Irracionais}
\date{\vspace{-5ex}}

\begin{document}

\maketitle

\textbf{\begin{center}
    Professor: Rogério Tiúma
\end{center}}


\section*{Parte Te\'orica}

\subsection*{Equa\c{c}\~oes do 2\textordfeminine\ Grau}

Uma equa\c{c}\~ao do segundo grau possui a forma geral:
\begin{equation}
ax^2 + bx + c = 0,
\end{equation}

onde $a$, $b$ e $c$ s\~ao n\'umeros reais, e $a \neq 0$.

\subsubsection*{F\'ormula de Bh\'askara}
As ra\'izes da equa\c{c}\~ao do segundo grau podem ser encontradas pela f\'ormula:
\begin{equation}
x = \frac{-b \pm \sqrt{\Delta}}{2a},
\end{equation}

em que o discriminante $\Delta$ \'e dado por:
\begin{equation}
\Delta = b^2 - 4ac.
\end{equation}

\textbf{Interpreta\c{c}\~ao do discriminante:}
\begin{itemize}
    \item Se $\Delta > 0$, existem duas ra\'izes reais e diferentes.
    \item Se $\Delta = 0$, existem duas ra\'izes reais e iguais.
    \item Se $\Delta < 0$, n\~ao existem ra\'izes reais (apenas ra\'izes complexas).
\end{itemize}

\subsubsection*{Fatora\c{c}\~ao}

Se a equa\c{c}\~ao puder ser escrita como um produto de dois bin\^omios, pode-se resolver usando fatorac\c{c}\~ao:
\begin{equation}
(x - r_1)(x - r_2) = 0,
\end{equation}

em que $r_1$ e $r_2$ s\~ao as ra\'izes da equa\c{c}\~ao.

\subsection*{N\'umeros Irracionais}

\textbf{Defini\c{c}\~ao:} Um n\'umero irracional \'e um n\'umero que n\~ao pode ser expresso como uma raz\~ao $\frac{p}{q}$ de dois n\'umeros inteiros ($p$, $q \neq 0$).

\textbf{Exemplos:}
\begin{itemize}
    \item $\sqrt{2}$, $\sqrt{3}$, $\pi$, $e$ (n\'umero de Euler).
\end{itemize}

\subsubsection*{Propriedades}
\begin{itemize}
    \item A raiz quadrada de um n\'umero inteiro positivo que n\~ao seja um quadrado perfeito \'e irracional.
    \item A soma ou a diferen\c{c}a entre um n\'umero racional e um irracional \'e irracional.
    \item O produto ou quociente entre um n\'umero racional diferente de zero e um irracional \'e irracional.
\end{itemize}

\section*{Lista de Exerc\'icios}

\begin{enumerate}

\item Resolva a equa\c{c}\~ao: $x^2 - 5x + 6 = 0$.

\item Resolva a equa\c{c}\~ao: $2x^2 - 8x + 6 = 0$.

\item Encontre as ra\'izes da equa\c{c}\~ao: $x^2 + 4x + 3 = 0$.

\item Calcule as ra\'izes de $3x^2 + 2x - 1 = 0$.

\item Determine as solu\c{c}\~oes de $5x^2 - 20 = 0$.

\item Encontre o valor de $x$ em $x^2 = 7$.

\item Resolva a equa\c{c}\~ao: $4x^2 - 12x + 9 = 0$.

\item Verifique se $\sqrt{2}$ \'e um n\'umero racional ou irracional.

\item Classifique o n\'umero $\sqrt{18}$ como racional ou irracional.

\item Simplifique a raiz $\sqrt{50}$.

\item Resolva a equa\c{c}\~ao: $x^2 + 2x - 15 = 0$.

\item Determine o valor de $x$ tal que $x^2 = 5$.

\item Diga se o n\'umero $\sqrt{8}$ \'{e} racional ou irracional e simplifique a raiz.

\item Resolva: $3x^2 - 2x - 1 = 0$.

\item Resolva a equa\c{c}\~ao: $x^2 - 2x + 1 = 0$.

\item Determine as ra\'izes de $x^2 + 1 = 0$ no conjunto dos n\'umeros reais.

\item Calcule o valor de $x$ sabendo que $2x^2 = 18$.

\item Resolva: $x^2 = 2$.

\item Classifique o n\'umero $\sqrt{0,25}$ como racional ou irracional.

\item Resolva a equa\c{c}\~ao $x^2 - 7x + 12 = 0$.

\end{enumerate}

\section*{Questões contextualizadas}

\begin{enumerate}

\item Um terreno retangular tem \(x\) metros de largura e \(x+4\) metros de comprimento. Sabendo que sua \'{a}rea \'e 60 m$^2$, encontre o valor de \(x\).

\item Em uma corrida, a dist\^{a}ncia percorrida em metros \'e dada por \(s = t^2 + 2t\), onde \(t\) \'e o tempo em segundos. Para que valor de \(t\) a dist\^{a}ncia ser\'a 35 metros?

\item Um fabricante deseja construir uma embalagem no formato de paralelep\'ipedo, onde a altura \(h\) em cm satisfa\c{c}a \(h^2 - 5h + 6 = 0\). Determine as poss\'iveis alturas.

\item Um n\'umero real positivo tem raiz quadrada igual \`a diferen\c{c}a entre 5 e o pr\'oprio n\'umero. Determine o n\'umero.

\item Uma piscina de forma retangular possui \(x\) metros de largura e \(x-2\) metros de comprimento. Se a \'{a}rea \'e 48 m$^2$, determine as dimens\~oes.

\item Um carpinteiro deseja construir uma moldura de quadro, onde a \'{a}rea interna deve ser de 30 cm$^2$, com lados medindo \(x\) e \(x+3\) cm. Determine \(x\).

\item Um investidor aplicou uma quantia e, ap\'os dois anos, seu capital \(C\) obedeceu \`a equa\c{c}\~ao \(C^2 - 100C + 2400 = 0\). Quanto foi aplicado inicialmente?

\item O per\'imetro de um tri\^angulo ret\^angulo \'e formado por catetos medindo \(x\) e \(x+2\) metros. Sabendo que a hipotenusa mede \(2\sqrt{5}\) metros, determine \(x\).

\item Classifique e simplifique: \(\sqrt{18}\).

\item Em uma trilha ecol\'ogica, a dist\^{a}ncia \(d\) percorrida em quil\^ometros ap\'os \(t\) horas obedece \`a equa\c{c}\~ao \(d = t^2 + t\). Determine o tempo necess\'ario para percorrer 6 km.

\item Um cano tem formato parab\'olico e sua equa\c{c}\~ao de altura \(h\) em fun\c{c}\~ao da dist\^{a}ncia \(x\) \'e \(h = -x^2 + 4x\). Determine os valores de \(x\) onde a altura \'e zero.

\item Um bal\~ao sobe conforme a equa\c{c}\~ao \(h = 3t^2 + 6t + 2\), onde \(h\) \'e altura em metros e \(t\) \'e o tempo em segundos. Quando o bal\~ao atingir 20 metros de altura?

\item Simplifique \(\sqrt{50}\) e classifique como racional ou irracional.

\item A diferen\c{c}a entre um n\'umero e seu quadrado \'e 20. Qual o n\'umero?

\item Uma folha de papel tem \(x\) cm de largura e \(x+5\) cm de comprimento. Se a \'{a}rea \'e 84 cm$^2$, determine \(x\).

\item Em uma regi\~ao, o pre\c{c}o \(p\) da gasolina varia conforme \(p = 2x^2 + 3x\). Se o pre\c{c}o for R\$10,00, qual o valor de \(x\)?

\item Resolva a equa\c{c}\~ao \(2x^2 - 8x + 5 = 0\) usando a f\'ormula de Bh\'askara.

\item Determinar se \(\sqrt{7}\) \'e racional ou irracional.

\item O dobro do quadrado de um n\'umero menos o triplo do n\'umero \'e igual a 20. Determine o n\'umero.

\item Um jardineiro quer construir um canteiro quadrado. Sabendo que a \'{a}rea \'e 98 m$^2$, qual deve ser a medida de cada lado?


\end{enumerate}

\end{document}

